% examples: http://www.scomminc.com/pp/acmsig/ACM-SIG-workshop-overviews.pdf
%%
%% This is file `sample-sigconf.tex',
%% generated with the docstrip utility.
%%
%% The original source files were:
%%
%% samples.dtx  (with options: `sigconf')
%% 
%% IMPORTANT NOTICE:
%% 
%% For the copyright see the source file.
%% 
%% Any modified versions of this file must be renamed
%% with new filenames distinct from sample-sigconf.tex.
%% 
%% For distribution of the original source see the terms
%% for copying and modification in the file samples.dtx.
%% 
%% This generated file may be distributed as long as the
%% original source files, as listed above, are part of the
%% same distribution. (The sources need not necessarily be
%% in the same archive or directory.)
%%
%%
%% Commands for TeXCount
%TC:macro \cite [option:text,text]
%TC:macro \citep [option:text,text]
%TC:macro \citet [option:text,text]
%TC:envir table 0 1
%TC:envir table* 0 1
%TC:envir tabular [ignore] word
%TC:envir displaymath 0 word
%TC:envir math 0 word
%TC:envir comment 0 0
%%
%%
%% The first command in your LaTeX source must be the \documentclass command.
\documentclass[sigconf]{acmart}
\copyrightyear{2022}
\acmYear{2022}
\setcopyright{rightsretained}
\acmConference[SIGMOD '22]{Proceedings of the 2022 International Conference on Management of Data}{June 12--17, 2022}{Philadelphia, PA, USA}
\acmBooktitle{Proceedings of the 2022 International Conference on Management of Data (SIGMOD '22), June 12--17, 2022, Philadelphia, PA, USA}
\acmDOI{10.1145/3514221.3524078}
\acmISBN{978-1-4503-9249-5/22/06}
% Authors, replace the red X's with your assigned DOI string during the rightsreview eform process.

\settopmatter{printacmref=true}
\begin{document}
\fancyhead{}

%%
%% \BibTeX command to typeset BibTeX logo in the docs
%% Rights management information.  This information is sent to you
%% when you complete the rights form.  These commands have SAMPLE
%% values in them; it is your responsibility as an author to replace
%% the commands and values with those provided to you when you
%% complete the rights form.


%%
%% Submission ID.
%% Use this when submitting an article to a sponsored event. You'll
%% receive a unique submission ID from the organizers
%% of the event, and this ID should be used as the parameter to this command.
%%\acmSubmissionID{123-A56-BU3}

%%
%% The majority of ACM publications use numbered citations and
%% references.  The command \citestyle{authoryear} switches to the
%% "author year" style.
%%
%% If you are preparing content for an event
%% sponsored by ACM SIGGRAPH, you must use the "author year" style of
%% citations and references.
%% Uncommenting
%% the next command will enable that style.
%%\citestyle{acmauthoryear}

%%
%% end of the preamble, start of the body of the document source.

%%
%% The "title" command has an optional parameter,
%% allowing the author to define a "short title" to be used in page headers.
\title{DBTest'24: 10th International Workshop on Testing Database Systems}


%%
%% The "author" command and its associated commands are used to define
%% the authors and their affiliations.
%% Of note is the shared affiliation of the first two authors, and the
%% "authornote" and "authornotemark" commands
%% used to denote shared contribution to the research.
\author{Anja Gruenheid}
\affiliation{%
  \institution{Microsoft}
  \city{Zurich}
  \country{Switzerland}}
\email{anja.gruenheid@microsoft.com}

\author{Manuel Rigger}
\email{rigger@nus.edu.sg}
\affiliation{%
  \institution{National University of Singapore}
  %\city{Singapore}
  \country{Singapore}
}

%%
%% By default, the full list of authors will be used in the page
%% headers. Often, this list is too long, and will overlap
%% other information printed in the page headers. This command allows
%% the author to define a more concise list
%% of authors' names for this purpose.
%\renewcommand{\shortauthors}{Trovato and Tobin, et al.}

%%
%% The abstract is a short summary of the work to be presented in the
%% article.
\begin{abstract}
With the ever-increasing amount of data stored and processed and ever-evolving hardware technology, there is not only an ongoing need for testing database management systems but also data-intensive systems in general.
%Furthermore, emerging new technologies such as Non-Volatile Memory impose new challenges (e.g., avoiding persistent memory leaks and partial writes), and novel system designs targeting FPGAs, GPUs, and RDMA call for additional attention and sophistication to testing.
Reviving the previous success of nine previous workshops, the goal of the Workshop on Testing Database Systems (DBTest) 2024 is to bring researchers and practitioners from academia and industry together to discuss key problems and ideas related to testing database systems and applications.
The long-term objective of our efforts is to reduce the cost and time required to test and tune data management and processing products so that users and vendors can spend more time and energy on actual innovations.
\end{abstract}

%%
%% The code below is generated by the tool at http://dl.acm.org/ccs.cfm.
%% Please copy and paste the code instead of the example below.
%%
\begin{CCSXML}
<ccs2012>
   <concept>
       <concept_id>10002951.10002952.10003190</concept_id>
       <concept_desc>Information systems~Database management system engines</concept_desc>
       <concept_significance>500</concept_significance>
       </concept>
   <concept>
       <concept_id>10011007.10011074.10011099.10011102.10011103</concept_id>
       <concept_desc>Software and its engineering~Software testing and debugging</concept_desc>
       <concept_significance>500</concept_significance>
       </concept>
 </ccs2012>
\end{CCSXML}

\ccsdesc[500]{Information systems~Database management system engines}
\ccsdesc[500]{Software and its engineering~Software testing and debugging}
%%
%% Keywords. The author(s) should pick words that accurately describe
%% the work being presented. Separate the keywords with commas.
\keywords{testing of database management systems}

%% A "teaser" image appears between the author and affiliation
%% information and the body of the document, and typically spans the
%% page.
%\begin{teaserfigure}
%  \includegraphics[width=\textwidth]{sampleteaser}
%  \caption{Seattle Mariners at Spring Training, 2010.}
%  \Description{Enjoying the baseball game from the third-base
%  seats. Ichiro Suzuki preparing to bat.}
%  \label{fig:teaser}
%\end{teaserfigure}

%%
%% This command processes the author and affiliation and title
%% information and builds the first part of the formatted document.
\maketitle

\section{Workshop Motivation and Scope}
Database management systems are complex systems used in a variety of circumstances ranging from standalone secure instances, over distributed systems to deployments in the cloud, running diverse workloads such as traditional transactional or analytical workloads but also modern machine learning (ML) workloads, with a variety of features both generalized and specialized to use cases and hardware.
The evaluation and comparison of these systems is a non-trivial but crucially important problem to confirm their validity and success.
Testing and benchmarking are the de facto methods with which we can address this problem.
As such, testing allows us to verify the validity of systems and their features, i.e.,~whether they concisely and correctly address a data management problem, while benchmarking allows us to evaluate and compare the performance of these systems.

In industry as well as in research projects, we see a lot of different approaches to the testing and benchmarking of data management systems due to these system's inherent complexity and diversity.
For example, we have seen these approaches and the interest in data management testing methodology shift from monolithic systems to novel cloud settings, driven both by an increase and diversity in workloads as well as hardware such as FPGAs or GPUs.
These workloads as well as new classes of data-driven applications, i.e.,~machine learning and big data scenarios, represent new challenges in this space that have impacted how we think about the evaluation of data management systems and their performance.
This is just one direction that data management systems are developing into, at the same time, SQL and no-SQL engines alike are evolving, and new engine concepts such as unified engines are being created.
Consequentially, they require special testing efforts and rigor to ensure classical database strengths such as reliability, integrity, and performance can be successfully carried over to these novel system architectures and designs.

The goal of the 10th Workshop on Testing Database Systems (DBTest) co-located with SIGMOD is to bring together reseachers from academia and industry to discuss these diverse topics.
It allows researchers to discuss new approaches to testing and benchmarking, explore how they can be deployed in practice, and exchange novel insights and observations about existing systems.
Topics for the workshop include but are not limited to testing methodology for a variety of systems, frameworks, and technologies relevant for the database community, experience reports that discuss the implementation of testing, benchmarking, and performance evaluation platforms in industry, and how we as a community can make database performance and research reproducible.

\section{Categories of Papers}
\begin{itemize}
    \item {\bf Research Papers} propose new approaches, methodology and techniques related to testing and performance evaluation for data management systems.
    \item {\bf Experience Papers} showcase how testing strategies and approaches are applied in practicing, highlighting interesting and novel insights into how they can be effectively utilized and improved.
    \item {\bf Talk Proposals} describe a specific topic relevant to the workshop that the authors want to receive feedback on from the community. These proposals are not part of the proceedings of this workshop.
\end{itemize}

\section{Topics of Interest}
The workshop focuses on the following topics:
\begin{itemize}
    \item Testing and benchmarking of learning-based database systems
    \item Testing of database systems, storage services, and database applications
    \item Testing of database systems using novel hardware and software technology (non-volatile memory, hardware transactional memory, …)
    \item Testing heterogeneous systems with hardware accelerators (GPUs, FPGAs, ASICs, …)
    \item Testing distributed and big data systems
    \item Testing machine learning systems
    \item Testing the reliability and availability of database systems
    \item Performance and scalability testing
    \item Security and vulnerability testing
    \item Specific challenges of testing and quality assurance for cloud-based systems
    \item Tools for analyzing database management systems (e.g., profilers, debuggers)
    \item Operational aspects such as continuous integration and delivery pipelines
    \item Workload characterization with respect to performance metrics and engine components
    \item Metrics for test quality, robustness, efficiency, and effectiveness
    \item Testing the effectiveness of adaptive policies and components
    \item Algorithms and techniques for automatic program verification
    \item Maximizing code coverage during testing of database systems and applications
    \item Functional and performance testing of interactive data exploration systems
    \item Generation of synthetic data for test databases
    \item Tracability, reproducibility and reasoning for ML-based systems
    \item Reproducibility of database research
    \item Experimental reproduction of benchmark results
    \item War stories and lessons learned
\end{itemize}

\section{Committee}

We are grateful for the support from the steering committee as well as the hard work by the program committee.

\textbf{Steering Committee}
\begin{itemize}
	\item Carsten Binnig, TU Darmstadt
	\item Alexander Böhm, Google
	\item Tilmann Rabl, TU Berlin
\end{itemize}

\noindent{}\textbf{Program Committee}
\begin{itemize}
	\item Adam Dickinson (Snowflake)
	\item Amarnadh Sai Eluri (Google)
	\item Anupam Sanghi (IBM Research)
	\item Brian Kroth (Microsoft GSL)
	\item Danica Porobic (Oracle)
	\item Daniel Ritter (SAP)
	\item Jinsheng Ba (NUS)
	\item Renata Borovica-Gajic (University of Melbourne)
	\item Russell Sears (CrystalDB)
	\item Stefania Dumbrava (ENSIIE)
	\item Wensheng Dou (University of Chinese Academy of Sciences)
	\item Xiu Tang (Zhejiang University)
	\item Yannis Chronis (Google Research)
	\item Zuming Jiang (ETH)
\end{itemize}

\section{Keynote Speakers}
% http://people.cs.uchicago.edu/~shanlu/about/bio_ShanLu.html
\textbf{Everett (Ev) Maus (Staff Software Engineer, Google)}

Ev has a passion for teaching computers to find bugs --- ideally without making other developers too unhappy.
He's been on the Spanner Engineering Productivity team at Google in 2018, where he’s building tooling to sustainably ensure the correctness and reliability of Spanner. He was an invited industry attendee at the 2023 Daghstul Seminar on Ensuring the Reliability and Robustness of Database Management Systems.
Prior to Google, he worked at Microsoft from 2014 to 2018 on scaling static analysis and security tooling. While there, he contributed to the SARIF standard for tooling results and spoke at BlueHat about building security tooling.
When he isn’t trying to teach computers to break software (usually other people’s), he collects fountain pens, mixes a decent manhattan, plays violin, and recently adopted a puppy.
He graduated from the University of Virginia with a BA in Computer Science and Mathematics in December 2013.

\textbf{Jesús Camacho Rodríguez (Principal Research SDE Manager, Gray Systems Lab)}
Jesús is a Principal Research SDE Manager at the Gray Systems Lab (GSL), the applied research group within Azure Data. His research focuses broadly on optimizing data systems performance and efficiency, with close collaboration with product and engineering teams.
Before joining Microsoft, Jesús held various engineering positions at Cloudera, where he worked on query processing and optimization in Cloudera’s SQL data warehouse engines. He has also been actively involved with open-source projects such as Apache Calcite and Apache Hive.
Jesús earned his PhD from the University of Paris-Sud and Inria, France, and holds a degree in Computer Science and Engineering from the University of Almería, Spain.

\section{Format and Planned Timeline}

We intend for DBTest to be held as a full-day workshop within the arrangements of SIGMOD 2024 as an in-person event.
We have planned for the following preliminary timeline:
\begin{itemize}
    \item Submissions due: March 15, 2022 (postponed to March 8)
    \item Notification of outcome: April 11, 2024
    \item Camera-ready due: April 25, 2024
\end{itemize}


\end{document}
\endinput
%%
%% End of file `sample-sigconf.tex'.
