% examples: http://www.scomminc.com/pp/acmsig/ACM-SIG-workshop-overviews.pdf
%%
%% This is file `sample-sigconf.tex',
%% generated with the docstrip utility.
%%
%% The original source files were:
%%
%% samples.dtx  (with options: `sigconf')
%% 
%% IMPORTANT NOTICE:
%% 
%% For the copyright see the source file.
%% 
%% Any modified versions of this file must be renamed
%% with new filenames distinct from sample-sigconf.tex.
%% 
%% For distribution of the original source see the terms
%% for copying and modification in the file samples.dtx.
%% 
%% This generated file may be distributed as long as the
%% original source files, as listed above, are part of the
%% same distribution. (The sources need not necessarily be
%% in the same archive or directory.)
%%
%%
%% Commands for TeXCount
%TC:macro \cite [option:text,text]
%TC:macro \citep [option:text,text]
%TC:macro \citet [option:text,text]
%TC:envir table 0 1
%TC:envir table* 0 1
%TC:envir tabular [ignore] word
%TC:envir displaymath 0 word
%TC:envir math 0 word
%TC:envir comment 0 0
%%
%%
%% The first command in your LaTeX source must be the \documentclass command.
\documentclass[sigconf]{acmart}
\copyrightyear{2022}
\acmYear{2022}
\setcopyright{rightsretained}
\acmConference[SIGMOD '22] {Proceedings of the 2022 International Conference on Management of Data}{June 12--17, 2022}{Philadelphia, PA, USA.}
\acmBooktitle{Proceedings of the 2022 International Conference on Management of Data (SIGMOD '22), June 12--17, 2022, Philadelphia, PA, USA}
\acmPrice{}
\acmISBN{978-1-4503-9249-5/22/06}
\acmDOI{10.1145/XXXXXX.XXXXXX}
% Authors, replace the red X's with your assigned DOI string during the rightsreview eform process.

\settopmatter{printacmref=true}
\begin{document}
\fancyhead{}

%%
%% \BibTeX command to typeset BibTeX logo in the docs
%% Rights management information.  This information is sent to you
%% when you complete the rights form.  These commands have SAMPLE
%% values in them; it is your responsibility as an author to replace
%% the commands and values with those provided to you when you
%% complete the rights form.


%%
%% Submission ID.
%% Use this when submitting an article to a sponsored event. You'll
%% receive a unique submission ID from the organizers
%% of the event, and this ID should be used as the parameter to this command.
%%\acmSubmissionID{123-A56-BU3}

%%
%% The majority of ACM publications use numbered citations and
%% references.  The command \citestyle{authoryear} switches to the
%% "author year" style.
%%
%% If you are preparing content for an event
%% sponsored by ACM SIGGRAPH, you must use the "author year" style of
%% citations and references.
%% Uncommenting
%% the next command will enable that style.
%%\citestyle{acmauthoryear}

%%
%% end of the preamble, start of the body of the document source.

%%
%% The "title" command has an optional parameter,
%% allowing the author to define a "short title" to be used in page headers.
\title{DBTest ‘22: 9th International Workshop on Testing Database Systems}


%%
%% The "author" command and its associated commands are used to define
%% the authors and their affiliations.
%% Of note is the shared affiliation of the first two authors, and the
%% "authornote" and "authornotemark" commands
%% used to denote shared contribution to the research.
\author{Manuel Rigger}
\email{manuel.rigger@inf.ethz.ch}
\affiliation{%
  \institution{Department of Computer Science, ETH Zurich}
  \city{Zurich}
  \country{Switzerland}
}

\author{Pınar Tözün}
\affiliation{%
  \institution{IT University of Copenhagen}
  \city{Copenhagen}
  \country{Denmark}}
\email{pito@itu.dk}
%%
%% By default, the full list of authors will be used in the page
%% headers. Often, this list is too long, and will overlap
%% other information printed in the page headers. This command allows
%% the author to define a more concise list
%% of authors' names for this purpose.
%\renewcommand{\shortauthors}{Trovato and Tobin, et al.}

%%
%% The abstract is a short summary of the work to be presented in the
%% article.
\begin{abstract}
With the ever-increasing amount of data stored and processed and ever-evolving hardware technology, there is not only an ongoing need for testing database management systems but also data-intensive systems in general.
%Furthermore, emerging new technologies such as Non-Volatile Memory impose new challenges (e.g., avoiding persistent memory leaks and partial writes), and novel system designs targeting FPGAs, GPUs, and RDMA call for additional attention and sophistication to testing.
Reviving the previous success of the eight previous workshops, the goal of DBTest 2022 is to bring researchers and practitioners from academia and industry together to discuss key problems and ideas related to testing database systems and applications. The long-term objective is to reduce the cost and time required to test and tune data management and processing products so that users and vendors can spend more time and energy on actual innovations.
\end{abstract}

%%
%% The code below is generated by the tool at http://dl.acm.org/ccs.cfm.
%% Please copy and paste the code instead of the example below.
%%
\begin{CCSXML}
<ccs2012>
   <concept>
       <concept_id>10002951.10002952.10003190</concept_id>
       <concept_desc>Information systems~Database management system engines</concept_desc>
       <concept_significance>500</concept_significance>
       </concept>
   <concept>
       <concept_id>10011007.10011074.10011099.10011102.10011103</concept_id>
       <concept_desc>Software and its engineering~Software testing and debugging</concept_desc>
       <concept_significance>500</concept_significance>
       </concept>
 </ccs2012>
\end{CCSXML}

\ccsdesc[500]{Information systems~Database management system engines}
\ccsdesc[500]{Software and its engineering~Software testing and debugging}
%%
%% Keywords. The author(s) should pick words that accurately describe
%% the work being presented. Separate the keywords with commas.
\keywords{testing of database management systems}

%% A "teaser" image appears between the author and affiliation
%% information and the body of the document, and typically spans the
%% page.
%\begin{teaserfigure}
%  \includegraphics[width=\textwidth]{sampleteaser}
%  \caption{Seattle Mariners at Spring Training, 2010.}
%  \Description{Enjoying the baseball game from the third-base
%  seats. Ichiro Suzuki preparing to bat.}
%  \label{fig:teaser}
%\end{teaserfigure}

%%
%% This command processes the author and affiliation and title
%% information and builds the first part of the formatted document.
\maketitle

\section{Workshop Motivation and Scope}
We are running the 9th instance of the Workshop on Testing Database Systems (DBTest) at SIGMOD 2022.
After four years of absence, DBTest was successfully revived at SIGMOD 2018, with over 40 participants, 15 paper submissions (out of which 8 were accepted), and three industry sponsors (SAP, Snowflake, Undo).
Like previous instances, one of the remarkable aspects was the high number of industry participants and submissions, creating an interesting mix of both academic work and industry perspectives.
On the one hand, there was a lot of positive feedback and we were encouraged by the audience to re-establish DBTest in a yearly cadence.
On the other hand, in order to not repeat past mistakes and as announced in the 2018 workshop application, we decided to only run with a biannual cadence to prevent from running out of topics and high-quality submissions.
As a result, we applied and held the 8th instance for the DBTest workshop in 2020 in virtual format due to the pandemic with 10 submissions in total (out of which 5 were accepted as full paper and 2 were accepted as short poster papers).
We think the slightly lower number of submissions were due to the impact of the start of the pandemic.
Despite this, we received very high-quality submissions, and the workshop once again had a good mix of both academic and industry participants and presenters. In total, we had 150 registrants, and the three sessions had 72, 57, and 48 participants, respectively.
Consequentially, we decided to hold another instance of DBTest now for SIGMOD 2022.

We believe that there is the need and interest for discussing topics that are related to not only testing database management systems but also data-intensive systems in general.
Furthermore, emerging new technologies such as Non-Volatile Memory impose new challenges (e.g., avoiding persistent memory leaks and partial writes), and novel system designs targeting FPGAs, GPUs, and RDMA call for additional attention and sophistication to testing.
Moreover, there are new classes of data-driven applications (i.e., machine learning and big data scenarios) that need to be considered.
Another dimension includes completely new system designs---such as crowdsourcing applications, where testing imposes various monetary and functional challenges (i.e., result set quality and verification), scalable machine learning systems that combine distributed computing with modern hardware, or even cloud-based data management systems with high release cadence and extreme global reach.
Finally, there is an ever-increasing popularity and proliferation of NoSQL systems.
These systems are taking new approaches and design decisions (e.g. by replacing the traditional ACID guarantees with relaxed consistency models (BASE)) to increase performance and scalability.
Therefore, they require special testing efforts and rigor to ensure classical database strengths such as reliability, integrity, and performance can be successfully carried over to these novel system architectures and designs.

\section{Topics of Interest}
The workshop focuses on the following topics:
\begin{itemize}
    \item Testing of database systems, storage services, and database applications 
    \item Testing of database systems using novel hardware and software technology (non-volatile memory, hardware transactional memory, \ldots{})
    \item Testing heterogeneous systems with hardware accelerators (GPUs, FPGAs, ASICs, …)
    \item Testing distributed and big data systems
    \item Testing machine learning systems
    \item Specific challenges of testing and quality assurance for cloud-based systems 
    \item War stories and lessons learned
\end{itemize}
Other topics include:
\begin{itemize}
    \item Performance and scalability testing
    \item Testing the reliability and availability of database systems 
    \item Algorithms and techniques for automatic program verification 
    \item Maximizing code coverage during testing of database systems and applications 
    \item Generation of synthetic data for test databases 
    \item Testing the effectiveness of adaptive policies and components 
    \item Tools for analyzing database management systems (e.g., profilers, debuggers)
    \item Workload characterization with respect to performance metrics and engine components 
    \item Metrics for test quality, robustness, efficiency, and effectiveness 
    \item Operational aspects such as continuous integration and delivery pipelines
    \item Security and vulnerability testing
\end{itemize}

\section{Committee}

We are grateful for the support from the steering committee as well as the hard work by the program committee.

\textbf{Steering Committee}
\begin{itemize}
	\item Carsten Binnig, TU Darmstadt
	\item Alexander Böhm, Google
	\item Tilmann Rabl, TU Berlin
\end{itemize}

\noindent{}\textbf{Program Committee}
\begin{itemize}
	\item Anisoara Nica, SAP SE
	\item Anja Grünheid Microsoft
	\item Chee-Yong Chan, National University of Singapore
	\item Danica Porobic, Oracle
	\item Daniel Ritter, HPI
	\item Jayant R. Haritsa, Indian Institute of Science
	\item Joy Aruljay, Georgia Tech
	\item Junwen Yang, University of Chicago
	\item Muhammad Ali Gulzar, Virginia Tech
	\item Numair Mansur, MPI-SWS
	\item Renata Borovica-Gajic, University of Melbourne
	\item Shuai Wang, HKUST
	\item S. Sudarshan, IIT Bombay
	\item Stefania Dumbrava, ENSIIE
	\item Utku Sirin, Harvard University
\end{itemize}

\section{Keynote Speakers}
% http://people.cs.uchicago.edu/~shanlu/about/bio_ShanLu.html
\textbf{Shan Lu} is a Professor in the Department of Computer Science at the University of Chicago.
She received her Ph.D. at University of Illinois, Urbana-Champaign, in 2008. She was the Clare Boothe Luce Assistant Professor of Computer Sciences at University of Wisconsin, Madison, from 2009 to 2014. Her research focuses on software reliability and efficiency, particularly detecting, diagnosing, and fixing functional and performance bugs in large software systems.
Shan is an ACM Distinguished Member (2019 class), an Alfred P. Sloan Research Fellow (2014), a Distinguished Educator Alumnus from Department of Computer Science at University of Illinois (2013), and NSF Career Award recipient (2010).
Her co-authored papers won Google Scholar Classic Paper 2017, Best Paper Awards at ACM-SIGOPS SOSP 2019, USENIX OSDI 2016 and USENIX FAST 2013, 3 ACM-SIGSOFT Distinguished Paper Awards at ICSE 2019, ICSE 2015 and FSE 2014, an ACM CHI Honorable Mention Award 2021, an ACM-SIGPLAN Research Highlight Award at PLDI 2011, and an IEEE Micro Top Picks in ASPLOS 2006. Shan is also a member of the informal ASPLOS Hall of Fame.

\textbf{Mark Raasveldt} is the co-founder of DuckDB Labs, where he currently works as CTO and lead developer on the DuckDB database system. He is also working as a postdoc in the Database Architectures group at the Centrum Wiskunde \& Informatica (CWI). Mark did his PhD at the CWI, working on efficient integration of machine learning and analytics programs with relational database management systems. 

\section{Format and Planned Timeline}

We intend for DBTest to be held as a full-day workshop within the arrangements of SIGMOD 2022 as a hybrid event.
We have planned for the following preliminary timeline:
\begin{itemize}
    \item Submissions due: March 1, 2022 (postponed to March 11)
    \item Notification of outcome: April 1, 2022
    \item Camera-ready due: April 17, 2022
\end{itemize}


\end{document}
\endinput
%%
%% End of file `sample-sigconf.tex'.
